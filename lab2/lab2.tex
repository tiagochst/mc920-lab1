\documentclass[10pt,a4paper]{article}
\usepackage[T1]{fontenc}
\usepackage[brazil]{babel}
\usepackage[utf8]{inputenc}


\usepackage{ae,aecompl}
\usepackage{pslatex}
\usepackage{epsfig}
\usepackage{geometry}
\usepackage{url}
\usepackage{textcomp}
\usepackage{ae}
\usepackage{subfig}
\usepackage{indentfirst}
\usepackage{textcomp}
\usepackage{color}
\usepackage{setspace}
\usepackage{verbatim}
\usepackage{mathtools}
\usepackage{amsmath}


\usepackage[compact]{titlesec}
\titlespacing{\section}{0pt}{*0}{*0}
\titlespacing{\subsection}{0pt}{*0}{*0}
\titlespacing{\subsubsection}{0pt}{*0}{*0}

\linespread{1.5}
\geometry{ 
  a4paper,	% Formato do papel
  tmargin=25mm,	% Margem superior
  bmargin=25mm,	% Margem inferior
  lmargin=20mm,	% Margem esquerda
  rmargin=20mm,	% Margem direita
  footskip=10mm	% Espaço entre o rodapé e o fim do texto
}
\include{abaco} 
\renewcommand{\thetable}{\Roman{table}}
\newcommand{\x} {$\bullet$}


\begin{document}
% CAPA
\begin{titlepage}
  \thispagestyle{empty}
  \begin{center} {\large \textbf{UNIVERSIDADE~ESTADUAL~DE~CAMPINAS}} \end{center}
  \begin{center} {\large INSTITUTO~DE~COMPUTAÇÃO}                    \end{center}
  \vspace{0.1cm}
  \begin{center}
    \begin{minipage}[tl]{31mm}
      \ABACO{1}{9}{6}{9}{1}
    \end{minipage}
  \end{center}
  \vspace{0.3cm}
  \begin{center} 
    {\large \textsc{"Detecção de padrões de legendas em imagens de ritmo visual a
partir do detector de Harris"?  }} 
    \\\vspace{0.5cm}
    {\textsl{Relatório do segundo de MC920}}
    \\\vspace{1cm}
    \begin{tabular}{rl}
      \textbf{Aluno}:   Carlos~Eduardo~Rosa~Machado &
      \textbf{RA}:          059582 \\ 
      \textbf{Aluno}:        Tiago~Chedraoui~Silva & 
      \textbf{RA}:        082941 \\
      \textbf{Aluno}:        William~Marques~Dias & 
      \textbf{RA}:        065106 \\
    \end{tabular}
  \end{center}
  \vspace{0.5cm}

  \begin{abstract}
  \end{abstract}
  % Sumário
  \tableofcontents
\end{titlepage} 

\vspace{2mm}
\newpage

\section{Introdução}
\section{Métodos}

%corner = canto O.o
Desenvolveu-se em python~\cite{python} um programa para aplicar o
detector de cantos de Moravec.

Dado uma imagem $I$ retorna-se a imagem com os cantos realçados.
Para isso aplica-se a fórmula

\begin{equation}
E_{x,y}=\sum_{u,v}w_{u,v}\left | I_{x+u,y+v}-I_{u,v}  \right |^2
\end{equation}


\section{Comparação de imagens}

\section{Resultados}

\section{Conclusão}
% Necessária?
% \section{Agradecimentos}

% ******************************************************
% REFERENCIAS BIBLIOGRÁFICAS
% ******************************************************
% \section{Referências}
\bibliographystyle{plain}
\begin{small}
  \bibliography{referencias}
\end{small}

\end{document}
