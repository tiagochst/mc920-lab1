\documentclass[10pt,a4paper]{article}
\usepackage[T1]{fontenc}
\usepackage[brazil]{babel}
\usepackage[utf8]{inputenc}


\usepackage{ae,aecompl}
\usepackage{pslatex}
\usepackage{epsfig}
\usepackage{geometry}
\usepackage{url}
\usepackage{textcomp}
\usepackage{ae}
\usepackage{subfig}
\usepackage{indentfirst}
\usepackage{textcomp}
\usepackage{color}
\usepackage{setspace}
\usepackage{verbatim}
\usepackage{mathtools}
\usepackage{amsmath}


\linespread{1.5}
\geometry{ 
	a4paper,	% Formato do papel
	tmargin=30mm,	% Margem superior
	bmargin=20mm,	% Margem inferior
	lmargin=20mm,	% Margem esquerda
	rmargin=20mm,	% Margem direita
	footskip=20mm	% Espaço entre o rodapé e o fim do texto
}
\include{abaco} 
\renewcommand{\thetable}{\Roman{table}}
\newcommand{\x} {$\bullet$}


\begin{document}
% CAPA
\begin{titlepage}
\thispagestyle{empty}
  \begin{center} {\large \textbf{UNIVERSIDADE~ESTADUAL~DE~CAMPINAS}} \end{center}
  \begin{center} {\large INSTITUTO~DE~COMPUTAÇÃO}                    \end{center}
  \vspace{0.1cm}
  \begin{center}
  \begin{minipage}[tl]{31mm}
    \ABACO{1}{9}{6}{9}{1}
  \end{minipage}
  \end{center}
  \vspace{0.3cm}
  \begin{center} 
    {\large \textsc{São totalmente válidas algumas das considerações sobre a
Correlação de Pearson presentes na literatura?  }} 
    \\\vspace{0.5cm}
    {\textsl{Relatório do primeiro laboratório de MC920}}
    \\\vspace{1cm}
    \begin{tabular}{rl}
	  \textbf{Aluno}:   Carlos~Eduardo~Machado &  
	  \textbf{RA}:          059582 \\ 
	  \textbf{Aluno}:        Tiago~Chedraoui~Silva & 
	  \textbf{RA}:        082941 \\
	  \textbf{Aluno}:        William~Marques~Dias & 
	  \textbf{RA}:        065106 \\
	\end{tabular}
  \end{center}
  \vspace{0.5cm}

  \begin{abstract}
O coeficiente de correlação de Pearson é amplamente usado para
comparar imagens, contudo ele apresenta sérias limitações. Esse
trabalho consistiu na validação da análise realizada no Artigo ''The
Ineffectiveness of the Correlation Coefficient for Image Comparisons''.
  \end{abstract}
  % Sumário
  \tableofcontents
\end{titlepage} 

\vspace{2mm}
\newpage

\section{Introdução}
O coeficiente de correlação de Pearson é amplamente amplamente
utilizado na análise estatística, reconhecimento de padrões e
processamento de imagens.
Na área de processamento de imagens ele é utilizado na comparação de
duas imagens para fins de registro de imagens, reconhecimento de
objetos, e medição disparidade. Para imagens digitais monocromáticas, a correlação de Pearson é definido como :

\begin{equation}r = \frac{  
\displaystyle{\sum_{i} (x_i-\bar{x})(y_i-  
\bar{y})}}{\displaystyle{\left[  
\sum_{i}(x_i-\bar{x})^2  
\sum_{i}(y_i-\bar{y})^2\right]^{1/2}}}  
\end{equation}  

Onde $x_i$ é a intensidade dos pixels na imagem 1, $y_i$ é a
intensidade dos pixels na imagem 2,$\bar{x}$ é a intensidade média da
imagem 1 e $\bar{y}$ é a intensidade média da imagem 2.

O coeficiente tem valor $r=1$ se as duas imagens são absolutamente
idênticas, $r=0$ se são totalmente não correlacionadas e $r=-1$ se elas
são totalmente anti-correlacionadas, por exemplo se uma imagem é o
negativo da outra.

O coeficiente de Pearson pode ser utilizado em
aplicações de segurança, como, por exemplo, vigilância. Normalmente, 
o coeficiente de relação é usado para comparar duas imagens do mesmo
objeto ou cena durante vários momentos diferentes.

O objetivo deste trabalho é apresentar, através de exemplos, o
desempenho do coeficiente para comparações de imagens, especialmente
em aplicações de segurança.

\section{Métodos}



\subsection{Código}

\section{Resultados}

\section{Conclusão}

% ******************************************************
% 		REFERENCIAS BIBLIOGRÁFICAS
% ******************************************************
%\section{Referências}
\bibliographystyle{plain}
\begin{small}
  \bibliography{referencias}
\end{small}

\end{document}
